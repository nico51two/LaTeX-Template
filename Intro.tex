\section{Introduction}
\label{sec:Int}
\subsection{What this document is}
This document is intended to serve as a guideline for designing university assignment papers. It's purpose is
\begin{itemize}
  \item to teach myself \LaTeX 
  \item to develop a workflow (i.e. decide on resolutions to use when exporting graphics from R or maps from Qgis and catalog example images with known dimensions in this document)
  \item to provide an outline I can fill with contents
  \item to act as a reference for often used typesetting operations like inserting maps, tables, code blocks, etc.
  \item to standardize the visual appearance of my papers (in regards to color scheme, layout, typeface, etc.)
\end{itemize}
\subsection{Source code}
Because this document is at its core a template to kick-start my layout and formatting process, much of the information is actually found within the \textit{document code} (and not in the rendered document) either as comments in between lines of \LaTeX code or simply as preset \LaTeX blocks and sections that only need to be populated with content and compiled. If you are reading this text in a PDF viewer you might want to look at the code that was used to make it. You can download it from my gitHub repository:

\url{https://github.com/nico51two/LaTeX-Template}

or read it on \textit{overleaf.com} where I'll be working on the document until I find it necessary to start using an offline editor:

\url{https://de.overleaf.com/read/xdschdggxbzx}
\subsection{What this document isn't}
This is not a guide on how to use \LaTeX. It's not didactical and it is certainly not comprehensive. It's purely tailored around my individual needs and preferences. E.g. as of now I have not found it necessary to extensively comment the document code. This might change at some point when I start implementing more complex structures.